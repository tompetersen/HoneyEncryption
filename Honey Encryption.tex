%!TEX encoding = UTF-8 Unicode
\documentclass[12pt]{scrartcl}
\usepackage[utf8]{inputenc} % Unicode funktioniert unter Windows, Linux und Mac
\usepackage[T1]{fontenc}
\usepackage{ngerman}
\usepackage{graphicx}
\usepackage{amsmath}
\usepackage{hyperref}\urlstyle{rm}
\usepackage{times}
\usepackage[scaled]{helvet}
\usepackage{a4wide}
\usepackage{rotating}

\usepackage{listings}\lstset{breaklines=true,breakatwhitespace=true,frame=leftline,captionpos=b,xleftmargin=6ex,tabsize=4,numbers=left,numberstyle=\ttfamily\footnotesize,basicstyle=\ttfamily\footnotesize}
\sloppy
\setlength{\parindent}{0em}
\setlength{\parskip}{1.2ex plus 0.5ex minus 0.5ex}
\pagestyle{plain}

\begin{document}

\newpage
\thispagestyle{empty}
\begin{center}\Large
Universität Hamburg \par
Fachbereich Informatik
\vfill
{\Large\textsf{\textbf{Hinweise für das Erscheinungsbild von Seminar-, Studien- und Bachelor-, Master- und Diplomarbeiten}}\par} 
\bigskip
am Arbeitsbereich Sicherheit in Verteilten Systemen (SVS) \par
\bigskip
Prof. Dr. Hannes Federrath \par
\bigskip
\today
\vfill
\vfill 
(Muster für das Deckblatt: siehe letzte Seite dieser Hinweise)
\end{center}

\newpage
\section*{Zusammenfassung}

Für den eiligen Leser sollen auf etwa einer halben, maximal einer Seite die wichtigsten Inhalte, Erkenntnisse, Neuerungen bzw. Ergebnisse der Arbeit beschrieben werden. 

Durch eine solche Zusammenfassung (im engl. auch Abstract genannt) am Anfang der Arbeit wird die Arbeit deutlich aufgewertet. Hier sollte vermittelt werden, warum der Leser die Arbeit lesen sollte.

\newpage
\tableofcontents

\newpage
\section{Verschlüsselungsschema}

Nachdem eine Nachricht aus dem Message Space - wie im vorherigen Kapitel beschrieben - durch eine DTE auf den Seed Space abgebildet wurde, folgt die Verschlüsselung des Ergebnisses. Hierfür schlägt [\cite{EURO2014}] zwei verschiedene Vorgehensweisen vor. Die unter Verwendung dieser Vorgehensweisen entstehenden Honey Encryption-Schemata werden im Folgenden dargestellt.

\subsection{Hashbasierte Verschlüsselung}

Im ersten Schritt ...

\begin{figure}[h]
	\begin{align*}
		\text{HEnc}&_{\text{Hash}}(M, K)\\
		&S \overset{<r>}{=} \text{DTE}(M)\\ 	%Encoding
		&K_D = \text{PBKDF}(K)\\ 		%PBKDF2
		&R \overset{<r>}{=} {0,1}^n\\	%Random
		&H = \text{HF}(K_D,R)\\	%Hash
		&C = H \oplus S\\	%XOR
		&\text{Return } (C,R)
	\end{align*}
	\caption{Hashbasierte Verschlüsselung}
	\label{fig:HashEnc}
\end{figure}

\begin{figure}[h]
	\begin{align*}
		\text{HDec}&_{\text{Hash}}((C,R), K)\\
		&K_D = \text{PBKDF}(K)\\ 		%PBKDF2
		&H = \text{HF}(K_D,R)\\	%Hash
		&S = H \oplus C\\	%XOR
		&M = \text{DTE}^{-1}(S)\\ 	%Encoding
		&\text{Return } M
	\end{align*}
	\caption{Hashbasierte Entschlüsselung}
	\label{fig:HashDec}
\end{figure}



\subsection{Auf Blockchiffren basierte Verschlüsselung}

\begin{figure}[h]
	\begin{align*}
		\text{HEnc }&_{\text{Block}}(M, K)\\
		&S \overset{<r>}{=} \text{DTE}(M)\\ 	%Encoding
		&R \overset{<r>}{=} {0,1}^k\\	%Random
		&K' = \text{HF}(K,R)\\ 		%Hash
		\\
		&P = \epsilon \\
		&\text{For } i = 1 \text{ to } \left\lceil \frac{|S|}{n} \right\rceil \\
		&\qquad P = P \text{ || Enc}(K',i)\\	%Enc
		\\
		&C = P[1 .. |S|] \oplus S\\	%XOR
		&\text{Return } (C,R)
	\end{align*}
	\caption{Verschlüsselung mit Blockchiffre (CTR-Modus)}
	\label{fig:BlockEnc}
\end{figure}

\begin{figure}[h]
	\begin{align*}
		\text{HDec }&_{\text{Block}}((C,R), K)\\
		&K' = \text{HF}(K,R)\\ 		%Hash
		\\
		&P = \epsilon \\
		&\text{For } i = 1 \text{ to } \left\lceil \frac{|S|}{n} \right\rceil \\
		&\qquad P = P \text{ || Enc}(K',i)\\	%Enc
		\\
		&S = P[1 .. |S|] \oplus M\\	%XOR
		&M = \text{DTE}^{-1}(S)\\ 	%Encoding
		&\text{Return } M
	\end{align*}
	\caption{Entschlüsselung mit Blockchiffre (CTR-Modus)}
	\label{fig:BlockDec}
\end{figure}



\end{document}
