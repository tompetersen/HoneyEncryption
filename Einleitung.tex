\section{Einleitung}

Auf der EUROCRYPT 2014 wurde von Ari Juels und Thomas Ristenpart ein neues Verfahren zur Absicherung von Verschlüsselung unter Nutzung von schwachen Schlüsseln (wie nutzergewählten Passwörtern) vorgestellt, das die Autoren in Anlehnung an bereits bestehende, ähnliche Systeme als \emph{Honey Encryption} bezeichnet haben. 

Das Neuartige an diesem Verfahren ist, dass jede Entschlüsselung eines Ciphertexts unter einem zufälligen, nicht korrekten Schlüssel zu einer plausiblen Nachricht führt. Ein Angreifer, der keine weiteren Informationen über die Nachricht besitzt, kann die Nachricht so im Gegensatz zu bestehenden Verfahren nicht durch bloßes Ausprobieren aller möglichen Schlüssel knacken. Gerade nachdem in den letzten Jahren nach Angriffen auf Informationssysteme immer wieder Passwörter von Millionen von Nutzern bekannt geworden sind, erlangten  Angreifer so relativ klare Vorstellungen von der Passwortauswahl von Nutzern. Honey Encryption kann hier einen Fortschritt in der Sicherheit Passwort-basierter Verschlüsselung bieten.

In dieser Seminararbeit soll das Verfahren der Honey Encryption näher beleuchtet werden. Dazu wird in Abschnitt \ref{sec:funktionsweise} ein grober Überblick über heute verwendete Verfahren und ihre Schwächen gegeben und anhand eines Beispiels in die Funktionsweise und die notwendigen Einzelschritte der Honey Encryption eingegangen. Abschnitt \ref{sec:dte} und \ref{sec:schema} erklären diese Einzelschritte dann ausführlicher und erläutern das Vorgehen auch anhand von konkreten Beispielen. In Abschnitt \ref{sec:probleme} werden Einschränkungen betrachtet, die aus dem Verfahren selbst entstehen oder die bei seiner Anwendung beachtet werden müssen.

Da das Verfahren erst vor kurzer Zeit vorgestellt und noch keine relevante Sekundärliteratur zu diesem Thema veröffentlicht wurde, folgt diese Ausarbeitung in weiten Teilen der Argumentation von Juels und Ristenpart in \cite{IEEE2014} bzw. \cite{EURO2014}.
