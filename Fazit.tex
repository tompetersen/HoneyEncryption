\section{Fazit}
\label{sec:fazit}
Zusammenfassend lässt sich sagen, dass Honey Encryption mit ihrem neuen Ansatz, Nachrichten zu verschlüsseln, ein interessantes neues Forschungsfeld eröffnet. Sie ermöglicht es, auch kurze Schlüssel wie nutzergenerierte Passwörter zu verwenden und stellt mit ihrer Sicherheit einen Mehrwert dar.

Allerdings überwiegen im jetzigen Forschungszustand die Nachteile und Probleme (aufgezeigt in Abschnitt \ref{sec:probleme}). Die verlangte Kenntnis über die Menge aller Nachrichten und ihrer Wahrscheinlichkeitsverteilung verhindert die großflächige Anwendung von Honey Encryption. Ebenfalls sollte das Problem der \emph{Typo-Safety} weiter untersucht werden, um die Nutzererfahrung zu verbessern.

Die zukünftigen Forschungsbestreben sollten darauf abzielen, das Erstellen einer sicheren DTE zu vereinfachen. Ein Anfang wäre es, eine öffentlich zugängliche Sammlung von Honey Encryption Schemata für bestimmte Anwendungsfälle zur Verfügung zu stellen. Damit könnten Entwickler, die strukturierte Daten verschlüsseln wollen, auf diese zurückgreifen und müssten lediglich die Einbindung in ihr bestehendes System berücksichtigen. Eine immer größer werdende Sammlung führt somit zu weniger Aufwand für Entwickler und damit zu einer größeren und vor allem schnelleren Verbreitung des Verfahrens. Mit Hilfe modernerer Technik, statistischen Verfahren und weiteren Hilfsmitteln könnte es vielleicht sogar möglich sein, DTEs automatisiert generieren zu lassen. Die Probleme, die dabei auftreten und die Bedingungen, die an eine gute DTE gestellt werden, stehen der Entwicklung momentan noch im Weg. Allerdings gibt es Fortschritte in der Analyse und Generierung von natürlicher Sprache, die bei der Entwicklung der genannten Systeme behilflich sein können (\cite{CRCS2014}).

Ein weiteres Forschungsziel sollte es sein, die Möglichkeiten und Anwendungsbereiche von Honey Encryption zu erweitern. So könnte eine spannende Forschungsfrage sein, in wie weit es nicht doch möglich wäre, Klartexte zu verschlüsseln. Eine Idee könnte sein, die korrekte Nachricht als Grundlage für einen sehr viel größeren Message Space zu nutzen. Dabei würde eine Veränderung von Wörtern ohne Beeinträchtigung des Sinnzusammenhanges wichtig sein. Ein Beispiel wäre die Abwandlung des Satzes ``Der geheime Treffpunkt ist Hamburg'' in Nachrichten wie ``Der geheime Treffpunkt ist Berlin''. Der Sinngehalt bliebe der gleiche, ein potentieller Angreifer könnte dann nicht entscheiden, in welcher Stadt nun der \emph{geheime Treffpunkt} liegt. Die automatische Generierung solcher sinnverwandten Sätze wäre hier allerdings die erste Anlaufstelle, da der Message Space entsprechend groß gewählt werden muss. Dies ist so bei heutigem Kenntnisstand noch nicht möglich, allerdings könnte es in Zukunft solch ein Verfahren geben. Die Analyse der Sicherheit eines solchen Ansatzes wäre dann in einer weiteren wissenschaftlichen Arbeit zu klären.

Honey Encryption ist ein interessanter Ansatz, der noch viel Platz für Fortschritt und Verbesserung bietet. Die gebotene Sicherheit, die damit theoretisch möglich ist, sollte in der heutigen Zeit, mit immer wieder vorkommenden Passwort-Leaks und stärkeren Rechnern, Anreiz sein, weitere Forschung in dieser Richtung zu betreiben. Mithilfe immer besser funktionierender Sprachverarbeitung, komplexerer stochastischer Modelle und weiterer interdisziplinärer Erkenntnisse, wird Honey Encryption eventuell der Weg zu einer großflächigeren Anwendung geebnet.