% ===============================================================================
% = LaTeX Beamer Template des Arbeitsbereichs Sicherheit in verteilten Systemem
% = (c) 2012 Prof. Dr. Hannes Federrath, Uni Hamburg, Fachbereich Informatik
% = http://www.informatik.uni-hamburg.de/svs/
% ===============================================================================
%
\documentclass[t]{beamer} 
% Option t              Place text of slides at the (vertical) top of the slides.
% Option handout        Ein PDF ohne Pausen und Overlayeffekte erzeugen.
% Option aspectratio=43 169 => 16:9, 1610 => 16:10, 43 => 4:3
\usepackage[utf8]{inputenc}
\usepackage[ngerman]{babel}
\usepackage{graphicx,xcolor}
\usepackage{tikz}
\usepackage[T1]{fontenc} % 8-Bit-Zeichen; ermöglicht korrektes Kopieren von Umlauten aus dem pdf 
\usepackage[scaled]{helvet}

\usepackage{beamerthemedefault}

% Ränder definieren
\setbeamersize{text margin left=5ex, text margin right=5ex}

% Farbdefinitionen
\definecolor{svsgrau1}{RGB}{191,191,191} % Balken in Kopfzeile
\definecolor{svsgrau2}{RGB}{123,123,123} % Folienüberschriften
\definecolor{svsrot}{RGB}{255,0,0} % Bullets
\definecolor{svshellblau1}{RGB}{153,204,255} % Block Hintergrund
\definecolor{svshellblau2}{RGB}{24,113,248} % Anstrich Ebene 2
\definecolor{svsdunkelblau}{RGB}{38,82,128} % Text der Ebene 1

% Navigationsleiste ausblenden
\beamertemplatenavigationsymbolsempty 

% Farben der Bullets der Ebenen
\setbeamercolor{itemize item}{fg=svsrot}
\setbeamercolor{itemize subitem}{fg=svshellblau2}
\setbeamercolor{enumerate item}{parent=itemize item}
\setbeamercolor{enumerate subitem}{parent=itemize subitem}

% Formen der Bullets der Ebenen
\setbeamertemplate{itemize item}[circle] 
\setbeamertemplate{itemize subitem}{--} 
\setbeamertemplate{itemize subsubitem}[circle] 

% Farben der Texte 
\setbeamercolor{title}{fg=black}
\setbeamercolor{structure}{fg=svsgrau2}
\setbeamercolor{section in toc}{fg=black}
\setbeamercolor{framesubtitle}{fg=svsdunkelblau}
\setbeamercolor{itemize/enumerate body}{fg=svsdunkelblau}
\setbeamercolor{itemize/enumerate subbody}{fg=black}
\setbeamercolor{itemize/enumerate subsubbody}{fg=black}

% Zeichensätze der Texte
\setbeamerfont{author}{size=\normalsize}
\setbeamerfont{institute}{size=\normalsize}
\setbeamerfont{date}{size=\normalsize}
\setbeamerfont{frametitle}{size=\large}
\setbeamerfont{framesubtitle}{size=\footnotesize\raggedleft}
\setbeamerfont{sections/subsections in toc}{size=\normalsize}
\setbeamerfont{itemize/enumerate body}{size=\normalsize}
\setbeamerfont{itemize/enumerate subbody}{size=\normalsize}
\setbeamerfont{itemize/enumerate subsubbody}{size=\normalsize}

% Definitionen für farbig hinterlegten Block 
\setbeamertemplate{blocks}[rounded]
\setbeamercolor{block title}{fg=black,bg=svshellblau1}
\setbeamercolor{block body}{parent=normal text,use=block title,bg=block title.bg!25!bg}
\setbeamerfont{block title}{size=\normalsize}
\setbeamerfont{block body}{size=\normalsize}

% Definitionen für Agenda (FIXME: noch stärker an normale Listendefs. anpassen)
\setbeamertemplate{section in toc}[sections numbered]
\setbeamertemplate{subsection in toc}[square]

% Kopfzeile
\setbeamertemplate{headline}{
	\includegraphics[height=8mm]{pic/UHH-Logo_2010_ohneText.png}%
	\color{svsgrau1}\rule{\paperwidth}{8mm}\newline
	\mbox{}\rule{1em}{0pt}\rule{0pt}{8ex}
	\dotfill\newline\vspace{-7.3ex}
}

% Fusszeile
\setbeamertemplate{footline}[text line]{
	\parbox[b]{50mm}{\insertframenumber\\[1ex]}
}

% Hintergrund Titelseite
\defbeamertemplate{background canvas}{titlepage}{%
	{\color{svsgrau1}\vrule width\paperwidth height0.7\paperheight}%
	{\color{white}\vrule width\paperwidth height0.3\paperheight}%
}

% =============================
% = Ab hier Inhalte ändern... 
% =============================

\title{Funktionsweise der Honey Encryption}
\author{Konstantin Kobs\\Tom Petersen}
% \institute[Uni Hamburg]{Universität Hamburg\\ Fachbereich Informatik}
\date{20. Januar 2015}

\begin{document}

\begingroup
	\setbeamertemplate{background canvas}[titlepage]
	\begin{frame}[plain]
		\vskip8mm
		\includegraphics[width=2.2cm]{pic/svs_logo_hires-ohne-was.png}
		%\vskip-20mm % dies geht nur bei kurzen Vortragstiteln
		\titlepage
		\vspace{\fill}
		\includegraphics[width=2.9cm]{pic/UHH-Logo_2010_Farbe_RGB_hires_nomargin.png}
		\vskip20pt
	\end{frame}
\endgroup

\begin{frame}
	\tableofcontents
\end{frame}

\section{Einleitung}

\begin{frame}[c]
	\frametitle{Brute-Force-Angriff auf klassische Verfahren}

	\begin{align*}
		K_1 &\rightarrow \text{yxV\#U}\\
		K_2 &\rightarrow \text{Katze}\\
		K_3 &\rightarrow \text{-CPK9}\\
		\dots &\rightarrow \dots
	\end{align*}

\end{frame}

\begin{frame}[c]
	\frametitle{Verwendete Passwörter}

	\begin{center}
		\includegraphics[width=\textwidth]{pic/passwordscloud.png}
		\vspace*{.5cm}
	\end{center}
	\hfill \tiny{\emph{https://xato.net/wp-content/xup/passwordscloud.png}}
\end{frame}

\begin{frame}[c]
	\frametitle{Honey Encryption - Idee}

	\begin{align*}
		K_1 &\rightarrow \text{Hund}\\
		K_2 &\rightarrow \text{Katze}\\
		K_3 &\rightarrow \text{Maus}\\
		\dots &\rightarrow \dots
	\end{align*}
	
\end{frame}

\begin{frame}[c]
	\frametitle{Honey Encryption}

	\begin{quote}
		Honey Encryption wurde entwickelt, um Ciphertexte zu generieren, die bei Entschlüsselung mit einem falschen Schlüssel zu einem plausibel wirkenden, aber unechten Klartext führen.
	\end{quote}
	
	\vspace*{1cm}

	\hfill \textit{- A. Juels, T. Ristenpart}
\end{frame}

\section{Beispiel}

\begin{frame}[t]
	\frametitle{Beispiel}

\end{frame}

\section{Verfahren der Honey Encryption}
\subsection{Distribution Transforming Encoder}

\begin{frame}[t]
	\frametitle{DTE}

	\begin{figure}
	\center
	\begin{tikzpicture}[scale=0.6]
		% Linker Kasten
		\begin{scope}[opacity = 0.3]
			\draw (1, 8) rectangle ++ (3, 2) node [midway] {$00$};
			\draw (1, 6) rectangle ++ (3, 2) node [midway] {$01$};
			\draw (1, 4) rectangle ++ (3, 2) node [midway] {$10$};
			\draw (1, 2) rectangle ++ (3, 2) node [midway] {$11$};
			\node at (2.5, 1) {$\mathcal{K}$};
		\end{scope}
		% Mittlerer Kasten
		\begin{scope}
			\draw (7, 8) rectangle ++ (3, 2) node [midway] {$00$};
			\draw (7, 6) rectangle ++ (3, 2) node [midway] {$01$};
			\draw (7, 4) rectangle ++ (3, 2) node [midway] {$10$};
			\draw (7, 2) rectangle ++ (3, 2) node [midway] {$11$};
			\node at (8.5, 1) {$\mathcal{S}$};
		\end{scope}
		% Rechter Kasten
		\begin{scope}
			\draw (13, 8) rectangle ++ (3, 2) node [midway] {r};
			\draw (13, 6) rectangle ++ (3, 2) node [midway] {g};
			\draw (13, 2) rectangle ++ (3, 4) node [midway] {b};
			\node at (14.5, 1) {$\mathcal{M}$};
		\end{scope}
		% Linien
		\draw [color = red] (10, 9) --++ (3, 0);
		\draw [color = red] (10, 7) --++ (3, 0);
		\draw [color = red] (10, 5) --++ (3, -1);
		\draw [color = red] (10, 3) --++ (3, 1);
	\end{tikzpicture}
	\end{figure}

\end{frame}

\begin{frame}[t]
	\frametitle{DTE}

	$$DTE = (encode, decode)$$
	
	\begin{itemize}
		\item $encode$ meist randomisiert
		\item $decode$ deterministisch
	\end{itemize}
	
	\begin{figure}
	\center
	\begin{tikzpicture}[scale=0.3]
		% Mittlerer Kasten
		\begin{scope}
			\draw (7, 8) rectangle ++ (3, 2) node [midway] {$00$};
			\draw (7, 6) rectangle ++ (3, 2) node [midway] {$01$};
			\draw (7, 4) rectangle ++ (3, 2) node [midway] {$10$};
			\draw (7, 2) rectangle ++ (3, 2) node [midway] {$11$};
			\node at (8.5, 1) {$\mathcal{S}$};
		\end{scope}
		% Rechter Kasten
		\begin{scope}
			\draw (13, 8) rectangle ++ (3, 2) node [midway] {r};
			\draw (13, 6) rectangle ++ (3, 2) node [midway] {g};
			\draw (13, 2) rectangle ++ (3, 4) node [midway] {b};
			\node at (14.5, 1) {$\mathcal{M}$};
		\end{scope}
		% Linien
		\draw (10, 9) --++ (3, 0);
		\draw (10, 7) --++ (3, 0);
		\draw (10, 5) --++ (3, -1);
		\draw (10, 3) --++ (3, 1);
	\end{tikzpicture}
	\end{figure}	
	
\end{frame}

\begin{frame}
	\frametitle{DTE}
	Mögliche DTE-Formen:
	
	\begin{itemize}
		\item Tabelle/Datenstruktur zum Nachschauen
	\end{itemize}
	
	\center
	\begin{tabular}{|c|c|}
		\hline
		\textbf{Seed} & \textbf{Nachricht} \\
		\hline
		00 & rot\\
		\hline
		01 & grün\\
		\hline
		10, 11 & blau\\
		\hline
	\end{tabular}
	
	\begin{itemize}
		\item Funktion zur Berechnung
	\end{itemize}
		
			
\end{frame}

\begin{frame}[t]
	\frametitle{DTE}

	Bekannt sein muss:
	\begin{itemize}
		\item Menge/Struktur der Nachrichten
		\begin{itemize}
			\item endlich speicherbar/berechenbar
			\item unendlich berechenbar
		\end{itemize}
		\item Verteilung der Nachrichten
		\begin{itemize}
			\item Nachricht wahrscheinlicher $\Rightarrow$ mehr Seeds
		\end{itemize}
	\end{itemize}

	\begin{figure}
	\center
	\begin{tikzpicture}[scale=0.3]
		% Mittlerer Kasten
		\begin{scope}
			\draw (7, 8) rectangle ++ (3, 2) node [midway] {$00$};
			\draw (7, 6) rectangle ++ (3, 2) node [midway] {$01$};
			\draw (7, 4) rectangle ++ (3, 2) node [midway] {$10$};
			\draw (7, 2) rectangle ++ (3, 2) node [midway] {$11$};
			\node at (8.5, 1) {$\mathcal{S}$};
		\end{scope}
		% Rechter Kasten
		\begin{scope}
			\draw (13, 8) rectangle ++ (3, 2) node [midway] {r};
			\draw (13, 6) rectangle ++ (3, 2) node [midway] {g};
			\draw (13, 2) rectangle ++ (3, 4) node [midway] {b};
			\node at (14.5, 1) {$\mathcal{M}$};
		\end{scope}
		% Linien
		\draw (10, 9) --++ (3, 0);
		\draw (10, 7) --++ (3, 0);
		\draw (10, 5) --++ (3, -1);
		\draw (10, 3) --++ (3, 1);
	\end{tikzpicture}
	\end{figure}	
	
\end{frame}

\subsection{Verschlüsselung}

\begin{frame}[c]
	\frametitle{Verschlüsselung}

	\begin{figure}
	\center
	\begin{tikzpicture}[scale=0.6]
		% Linker Kasten
		\begin{scope}
			\draw (1, 8) rectangle ++ (3, 2) node [midway] {$00$};
			\draw (1, 6) rectangle ++ (3, 2) node [midway] {$01$};
			\draw (1, 4) rectangle ++ (3, 2) node [midway] {$10$};
			\draw (1, 2) rectangle ++ (3, 2) node [midway] {$11$};
			\node at (2.5, 1) {$\mathcal{K}$};
		\end{scope}
		% Mittlerer Kasten
		\begin{scope}
			\draw (7, 8) rectangle ++ (3, 2) node [midway] {$00$};
			\draw (7, 6) rectangle ++ (3, 2) node [midway] {$01$};
			\draw (7, 4) rectangle ++ (3, 2) node [midway] {$10$};
			\draw (7, 2) rectangle ++ (3, 2) node [midway] {$11$};
			\node at (8.5, 1) {$\mathcal{S}$};
		\end{scope}
		% Rechter Kasten
		\begin{scope}[opacity = 0.3]
			\draw (13, 8) rectangle ++ (3, 2) node [midway] {r};
			\draw (13, 6) rectangle ++ (3, 2) node [midway] {g};
			\draw (13, 2) rectangle ++ (3, 4) node [midway] {b};
			\node at (14.5, 1) {$\mathcal{M}$};
		\end{scope}
		% Linien
		\draw [opacity = 0.3] (10, 9) --++ (3, 0);
		\draw [opacity = 0.3] (10, 7) --++ (3, 0);
		\draw [opacity = 0.3] (10, 5) --++ (3, -1);
		\draw [opacity = 0.3] (10, 3) --++ (3, 1);
	\end{tikzpicture}
	\end{figure}

\end{frame}

\begin{frame}[c]
	\frametitle{Hashbasierte Verschlüsselung}
	
	\begin{columns}
		\column{.5\textwidth}
		Verschlüsselung
		\begin{align*}
			\text{HEnc}&_{\text{Hash}}(M, K)\\
			&S \overset{<r>}{=} \text{DTE}_{\text{encode}}(M)\\ 	%Encoding
			&R \overset{<r>}{=} \{0,1\}^k\\	%Random
			&H = \text{HF}(K,R)\\	%Hash
			&C = H \oplus S\\	%XOR
			&\text{Return } (C,R)
		\end{align*}
		
		\pause

		\column{.5\textwidth}
		Entschlüsselung
		\begin{align*}
			\text{HDec}&_{\text{Hash}}((C,R), K)\\
			&H = \text{HF}(K,R)\\	%Hash
			&S = H \oplus C\\	%XOR
			&M = \text{DTE}_{\text{decode}}(S)\\ 	%Decoding
			&\text{Return } M
		\end{align*}
	\end{columns}
\end{frame}

\section{Einschränkungen}

\begin{frame}[t]
	\frametitle{Einschränkungen der Honey Encryption}
	
	\begin{itemize}
		\item Freitext nicht möglich
		\begin{itemize}
			\item \emph{Menge der Nachrichten} unendlich groß
			\item Verteilung nicht bekannt
		\end{itemize}
		\pause
		\item Vorab bekannte Informationen
		\begin{itemize}
			\item Angreifer hat Zusatzinformationen $\Rightarrow$ Verifizierung des Ergebnisses
			\item Sicherheit der Verschlüsselung
		\end{itemize}
		\pause
		\item \emph{Typo-Safety}
		\begin{itemize}
			\item Tippfehler führt zu falschen Daten
			\item große Stärke $\Rightarrow$ große Schwäche
		\end{itemize}
	\end{itemize}	
	
\end{frame}

\section{Fazit}

\begin{frame}[t]
	\frametitle{Fazit}

\end{frame}

\end{document}