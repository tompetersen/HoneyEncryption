\section{Hashbasierte Verschlüsselung}

\begin{frame}[c]
	\frametitle{Hashbasierte Verschlüsselung}
	
	\begin{columns}
		\column{.5\textwidth}
		Verschlüsselung
		\begin{align*}
			\text{HEnc}&_{\text{Hash}}(M, K)\\
			&S \overset{<r>}{=} \text{DTE}_{\text{encode}}(M)\\ 	%Encoding
			&R \overset{<r>}{=} \{0,1\}^k\\	%Random
			&H = \text{HF}(K,R)\\	%Hash
			&C = H \oplus S\\	%XOR
			&\text{Return } (C,R)
		\end{align*}
		
		\pause

		\column{.5\textwidth}
		Entschlüsselung
		\begin{align*}
			\text{HDec}&_{\text{Hash}}((C,R), K)\\
			&H = \text{HF}(K,R)\\	%Hash
			&S = H \oplus C\\	%XOR
			&M = \text{DTE}_{\text{decode}}(S)\\ 	%Decoding
			&\text{Return } M
		\end{align*}
	\end{columns}
\end{frame}

\begin{frame}
	\frametitle{Verschlüsselung mit Blockchiffren}

	\begin{block}{Blockchiffren}
		Blockchiffren sind \emph{symmetrische} Verschlüsselungsverfahren, die Klartexte und Ciphertexte in Bitgruppen fester Länge (\emph{Blöcken}) bearbeiten.
	\end{block}

	\pause

	\begin{itemize}
		\item Können unter bestimmten Voraussetzungen ebenfalls für \emph{HE} genutzt werden.
		\item Nur bestimmte Betriebsmodi (CTR, CBC) sind geeignet.
		\item Es darf kein Padding benötigt werden.
	\end{itemize}

\end{frame}