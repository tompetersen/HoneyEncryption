\section{Funktionsweise}
\label{sec:funktionsweise}

Hier sollte noch viel viel mehr stehen...

\subsection{Verwendete Verfahren}

\subsubsection{XOR}
Die XOR-(Exklusiv-ODER-)Verknüpfung ist ein bitweiser Operator, der für zwei unterschiedliche Eingangsbits 1 ergibt und ansonsten 0. Seine besondere Bedeutung für die Kryptographie liegt in dem Zusammenhang \(K \oplus K = 0\) und somit \((M \oplus K) \oplus K = M\), dass heißt zweifache Verknüpfung von M mit dem Bitstring K ergibt wiederum M. Diese zweifache Verknüpfung lässt sich als Ver- und Entschlüsselung interpretieren, wie es beispielsweise beim OneTimePad geschieht \cite{Schneier2006}.%S.15-20

\subsubsection{Hashfunktion}
HF

\subsubsection{Blockchiffre}
Enc/Dec(?) \textit{mit DTE abstimmen}

\subsubsection{Password Based Key Derivation Function}
PBKDF(2)

\subsubsection{Betriebsmodi für Blockchiffren}
CTR, CBC

\subsection{Notationen}

In den folgenden Kapiteln, in denen näher auf die Funktionsweise der Honey Encryption eingegegangen wird, werden folgende Notationen verwendet:

\begin{description}

\item[\(\overset{<r>}{=}\)] steht für eine nicht-deterministische Zuweisung. Dies kann entweder komplett zufällig geschehen (wie bei Belegung von zufälligen Bitstrings) oder zumindestens vom Zufall mitbestimmt werden (wie bei der Kodierung durch die DTE).

\item[\(a \oplus b\)] steht für die XOR-Verknüpfung von a und b.

\item[\(S || P\)] steht für die Konkatenierung der Zeichenketten S und P.

\item[\(S{[1..n]}\)] steht für die Nutzung der ersten n Zeichen von S.

\item[\(\epsilon\)] steht für die leere Zeichenkette.

\end{description}

\newpage